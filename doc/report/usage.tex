\documentclass{jsarticle}

\usepackage[dvipdfmx]{graphicx}
\usepackage{multicol}
\usepackage{here}
\usepackage{geometry}
\geometry{left=25mm,right=25mm,top=20mm,bottom=20mm}
\usepackage{url}
\usepackage{listings,jlisting}

\lstset{%
  language={C},
  basicstyle={\small},%
  identifierstyle={\small},%
  commentstyle={\small\itshape},%
  keywordstyle={\small\bfseries},%
  ndkeywordstyle={\small},%
  stringstyle={\small\ttfamily},
  frame={tb},
  breaklines=true,
  columns=[l]{fullflexible},%
  numbers=left,%
  xrightmargin=0zw,%
  xleftmargin=3zw,%
  numberstyle={\scriptsize},%
  stepnumber=1,
  numbersep=1zw,%
  lineskip=-0.5ex%
}

\title{Neural Weaher Forecaster ―― AIによる天気予報}
\author{マイコンG班 13EC602 郭柏辰, 14EC004 飯田頌平, 14EC552 陳玉皓, 14EC602 劉宇航}

\begin{document}

  \maketitle
  
  \begin{multicols}{2}
  
    \section{概要}

      センサから温度・湿度・気圧を計測し、
      そのデータを人工知能(AI)が考察して天気予報を行います。

      いま、ディスプレイに表示されているのは、
      \begin{itemize}
        \item 降水確率[%]
        \item 気温[℃]
        \item 気圧[hPa]
        \item 湿度[%]
      \end{itemize}
      です。
      降水確率は、いまから7時間後までの予測を一気に行い、
      何時間後にもっとも降水確率が高くなるのか?という情報を表示しています。

      一番雨が振りそうな時点と、その確率が予測できれば、
      「今日は傘を持っていこうか?」と悩まなくてもよくなります。

    \section{使い方(一般の方向け)}

      何もしなくても大丈夫です。
      ひとりでに天気予報をしてくれます。

      具体的には、
      \begin{enumerate}
        \item 現在の気温・気圧・湿度の測定
        \item 測定データの記録
        \item 人工知能によるデータの解析
        \item ディスプレイへ表示
      \end{enumerate}
      といった作業をコンピュータで制御しています。
      数秒ごとに測定・記録・解析・表示の一連の流れを繰り返すため、
      常に最新の天気予報を行います。

    \section{使い方(作業者向け)}

      \subsection{自動セットアップ(デフォルト)}

        ラズパイ本体に電源ケーブルを挿して起動すると、
        自動的にスクリプトが立ち上がります。
        ディスプレイに予測結果が表示されれば成功です。

      \subsection{マニュアルセットアップ}

        自動セットアップに失敗する時は、手動でスクリプトを立ち上げます。
        起動後、ラズパイのコンソールにアクセスしてください。

        \begin{itemize}
          \item \begin{description} \item[ユーザ]デフォルト \end{description}
          \item \begin{description} \item[パスワード]デフォルト \end{description}
        \end{itemize}
        
        アクセスに成功したら、以下のスクリプトを実行してください。
        \begin{lstlisting}[caption=起動スクリプト,label=setup]
          $ cd /home/pi
          $ ./asahisai.sh
        \end{lstlisting}

    
  \end{multicols}

\end{document}
